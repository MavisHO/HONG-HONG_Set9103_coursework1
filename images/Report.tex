% #######################################
% ########### FILL THESE IN #############
% #######################################
\def\mytitle{Coursework Report}
\def\mykeywords{Python, Flask, Cats, Museum, Breeds, Ragdoll, Persian}
\def\myauthor{HONG HONG}
\def\contact{40333300@napier.ac.uk}
\def\mymodule{Advance Web Technology (SET09103)}
% #######################################
% #### YOU DON'T NEED TO TOUCH BELOW ####
% #######################################
\documentclass[10pt, a4paper]{article}
\usepackage[a4paper,outer=1.5cm,inner=1.5cm,top=1.75cm,bottom=1.5cm]{geometry}
\twocolumn
\usepackage{graphicx}
\graphicspath{{./images/}}
%colour our links, remove weird boxes
\usepackage[colorlinks,linkcolor={black},citecolor={blue!80!black},urlcolor={blue!80!black}]{hyperref}
%Stop indentation on new paragraphs
\usepackage[parfill]{parskip}
%% Arial-like font
\usepackage{lmodern}
\renewcommand*\familydefault{\sfdefault}
%Napier logo top right
\usepackage{watermark}
%Lorem Ipusm dolor please don't leave any in you final report ;)
\usepackage{lipsum}
\usepackage{xcolor}
\usepackage{listings}
%give us the Capital H that we all know and love
\usepackage{float}
%tone down the line spacing after section titles
\usepackage{titlesec}
%Cool maths printing
\usepackage{amsmath}
%PseudoCode
\usepackage{algorithm2e}

\titlespacing{\subsection}{0pt}{\parskip}{-3pt}
\titlespacing{\subsubsection}{0pt}{\parskip}{-\parskip}
\titlespacing{\paragraph}{0pt}{\parskip}{\parskip}
\newcommand{\figuremacro}[5]{
    \begin{figure}[#1]
        \centering
        \includegraphics[width=#5\columnwidth]{#2}
        \caption[#3]{\textbf{#3}#4}
        \label{fig:#2}
    \end{figure}
}

\lstset{
	escapeinside={/*@}{@*/}, language=C++,
	basicstyle=\fontsize{8.5}{12}\selectfont,
	numbers=left,numbersep=2pt,xleftmargin=2pt,frame=tb,
    columns=fullflexible,showstringspaces=false,tabsize=4,
    keepspaces=true,showtabs=false,showspaces=false,
    backgroundcolor=\color{white}, morekeywords={inline,public,
    class,private,protected,struct},captionpos=t,lineskip=-0.4em,
	aboveskip=10pt, extendedchars=true, breaklines=true,
	prebreak = \raisebox{0ex}[0ex][0ex]{\ensuremath{\hookleftarrow}},
	keywordstyle=\color[rgb]{0,0,1},
	commentstyle=\color[rgb]{0.133,0.545,0.133},
	stringstyle=\color[rgb]{0.627,0.126,0.941}
}

\thiswatermark{\centering \put(336.5,-38.0){\includegraphics[scale=0.8]{logo}} }
\title{\mytitle}
\author{\myauthor\hspace{1em}\\\contact\\Edinburgh Napier University\hspace{0.5em}-\hspace{0.5em}\mymodule}
\date{}
\hypersetup{pdfauthor=\myauthor,pdftitle=\mytitle,pdfkeywords=\mykeywords}
\sloppy
% #######################################
% ########### START FROM HERE ###########
% #######################################
\begin{document}
	\maketitle
	\begin{abstract}
	   In the recent years,more and more families are willing to keep pets at homes. And the pets play an important role in the families gradually. This web-app is for people who like cats and want to know more about them. The web-app includes several sections below: the different breeds of cats, meanings of their actions, and also their history.
		
	\end{abstract}
    
	\textbf{Keywords -- }{\mykeywords}

	\section{Introduction}
	 This website was built in order to let more people know more about cats, and arouse people to protect this lovely animal. It contains some useful functions to help users to find the information they want quickly.
	
	\section{Redirect and errors}
    The source code of this website has the redirect function. A route-"/abcd" has been set, if users enter this route after "localhost:5000", it will link to home page instead. The website also provide the error pages, it used the app.errorhandler decorator, it was set to return" Couldn't find the page you requested". Therefore, if the users enter the error route which is not included in the @app.route, the errorhandler function  will return the message.
    
    \figuremacro{h}{re.JPG}{}{ - The code of Redirecr and errorhandler}{1.0}
    
    \section{Responses}
    To implement the Responses function, "<a href="/route"></a>" has been used in many places. For example, it was placed on both side of the images and texts so that the users can link to the route by clicking the pictures or texts. And the @app.route can build URLs and addresses for the pages of the web-app. 
    
    \figuremacro{h}{rrag.JPG}{Respondses}{ - The code of an image with a link}{1.0}
    
    \section{Requests}
    The login page has a "login" button, and it used the request function to display a form when the users click the "login" button. By this function, this page should be able to use GET then display another page. But this part code may have some mistakes, the page does not implement this function.
    
    \figuremacro{h}{log1.JPG}{Requests}{ - A part of code of Requests}{1.0}
    
    \section{Components}
    \paragraph{This web-app has two main sections, navbar and Breeds. The navbar used HTML to build frame and used CSS to design. }
    \paragraph{By CSS, it is set up the margin, size, font and the color of background and so on. And for each colume, has been set up the color of text and other data to let them look more beautiful.}
    \paragraph{By HTML, it has been set up six buttons on. The users can link to six different pages directly by clicking these buttons. The first button is "HOME", whatever which page the user is in, they can return to the home page using the "HOME" button.}
    \paragraph{The second button is "Breed", in this page, people can see many images of various kinds of cats, if people click the images or the texts below the pictures, they will link to the corresponding pages. At the bottom of the page, has the text to introduce some places to buy pets and pet supplies or adopt pets. The name of these places have all have a link to their office websites. These websites will be opened in a new tab because the use of target="blank".}
     \figuremacro{h}{rbs.JPG}{Breeds}{ - The link of images and texts}{1.0}
    \paragraph{The "Login" button is a page which can collect the usename and password of people. This page also has a "login" button, this button should display a page that uses the data from the form, but it doesn't work. Because that,this button was linked to the home page.The last three buttons will link to their corresponding pages, which only have some images and some texts-Action, History and Contact.}
    
     \figuremacro{h}{nav.JPG}{Navbar}{ - The HTML and CSS code of navbar}{1.0}
    
    
	
    
	
\section{Conclusion}
To sum up, this web-app contains the Redirect, Responses and Requests functions,and it also have a navbar. The navbar can link to Breeds, Action, History and login page, it also has a "HOME"button to retern back to home page. The Breeds page has images and some text can also link to another page-some to the introduction of one breed(only the first two have complete introduction, others will display"To be continued...",because I do not have enough time to finish all) ,others to other websites. This web-app also completes the redirect function, when user input route"/abcd", the function will link to the home page. This web-app uses Python, HTML, CSS...to design and decorate. But because I can not use this all languages expertly.It is not complete enough and some functions are not implemented. To make it better, it need to have complete contents of breeds. And the login pages should have more functions, for example, users can add their name,email and interests to create their account. And therefore, the users can make friends with others if they have common interests. In addition, when they log in the web next time,the web-app should have a function to check if their username and possword are right. But before that, the problems of requests should be solved.
\bibliographystyle{ieeetr}

		
\end{document}